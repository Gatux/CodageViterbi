\documentclass[11pt]{article}

\usepackage[left=0.75in, right=0.75in, top=0.75in, bottom=0.75in]{geometry}
\usepackage{layout}
\usepackage{ucs}
\usepackage[latin1]{inputenc}
\usepackage[T1]{fontenc}
\usepackage{titlesec}
\usepackage{graphicx}
\usepackage{amssymb}
\usepackage{amsmath}
\usepackage{dsfont}
\usepackage{caption}
\usepackage{subcaption}
\usepackage{array}
\usepackage{stmaryrd}



\title{\textbf{TS226 - TP de codage canal}\\\\Encodage et d�codage de codes convolutifs}
\author{Thomas MARCHAL - Maxime PETERLIN - Gabriel VERMEULEN\\\\{ENSEIRB-MATMECA, Bordeaux}}
\date{12 juin 2014}


\begin{document}

\maketitle
			
\tableofcontents

\newpage

\section{Encodage de codes convolutifs}
	
	\subsection{Principe d'encodage d'une s�quence $\textbf{m} = [m_0, \ldots, m_{L-1}]$ de bits � transmettre}
	
		Soit $n$ la longueur du code convolutif binaire.\\
		Soient $g_1(D), \ldots, g_n(D)$ les $n$ polyn�mes g�n�rateurs � coefficients dans $\mathbb{F}_2$ servant � encoder la s�quence $\textbf{m} = [m_0, \ldots, m_{L-1}]$.
		On forme la matrice $G$ dont les lignes sont les coefficients de ces derniers. \\
		\\
		Soient $C(D)$ et $\textbf{m}(D)$ respectivement les s�quences de bits encod�e et � transmettre que l'on repr�sente sous forme polynomiale.\\
		\\
		La relation permettant d'encoder le message $\textbf{m}$ est la suivante :
		\begin{align}
			C(D) &= \textbf{m}(D)G = [\textbf{m}(D)g_1(D), \ldots, \textbf{m}(D)g_n(D)]\\
					 &= [C_1(D), \ldots, C_n(D)]
		\end{align}
		\\
		La s�quence envoy�e est alors
		\[
			\underbrace{C_{1,0} \cdots C_{n,0}}_{t = 0}\ \ldots\ \underbrace{C_{1,t} \cdots C_{n,t}}_{instant\ t}
		\]
	
	\subsection{M�moire M du codeur}
	
		La m�moire M du codeur est donn�e par le retard maximal des registres � d�calage.\\
		Math�matiquement, cela s'exprime de la mani�re suivante :
		\[
			M = \underset{i \in  \llbracket 1, \ldots, n \rrbracket}{max}\ deg(g_i)
		\]
		
	\subsection{Principe de fonctionnement de la fonction $codconv$}
	
		La fonction $codconv$ prend en param�tre le message � encoder \textbf{m}, ainsi qu'un vecteur $g$ compos� des polyn�mes g�n�rateurs associ�s au code sous forme octale.\\
		\\
		
	
\section{D�codage des codes convolutifs : algorithme de Viterbi}
\section{Performances}
	
		

			

\end{document}